%\ssr{Термины и определения}
\ssr{ТЕРМИНЫ И ОПРЕДЕЛЕНИЯ}

В настоящем отчете о НИР применяют следующие термины с соответствующими определениями

\noindent
Блокчейн-технология -- выстроенная по определённым правилам непрерывная последовательная цепочка блоков, содержащая информацию~\cite{blockchain}

\noindent
Отказ системы -- поведение системы, не удовлетворяющее ее спецификациям~\cite{lec}

\noindent
Распределенная система -- совокупность автономных вычислительных элементов, которая для его пользователей является единой связанной системой~\cite{Tanenbaum}

%\ssr{Перечень сокращений и обозначений}

\clearpage
\ssr{ВВЕДЕНИЕ}

При создании распределенных систем важным является организации взаимодействия ее одновременно работающих элементов между собой и принятия ими единого решения, то есть достижения консенсуса, даже в ситуации, когда некоторые узлы системы начинают работать неверно, злонамеренно или в случае поломки~\cite{distrDataProc}. Эта проблема формулируется как задача византийскиих генералов.

Целью данной работы является рассмотрение методов решения задачи византийских генералов.

Для достижения поставленной цели необходимо решить следующие задачи:
\begin{enumerate}[label={\arabic*)}]
	\item ввести основные определения;
	\item обозначить основные вехи развития;
	\item формализовать задачу византийских генералов;
	\item перечислить методы решения;
	\item сформулировать критерии сравнения;
	\item сравнить перечисленные методы по сформулированным критериям;
\end{enumerate}

\clearpage
