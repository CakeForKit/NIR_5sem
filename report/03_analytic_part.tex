
\chapter{Анализ предметной области}

В данном разделе описывается изначальная постановка и формализация задачи византийских генералов, также рассматривается область ее применения.

\section{Формализация задачи}

Задача византийских генералов была сформулирована в 1982 году в исследовании Л. Лэмпорт, Р. Шостак и М. Пиз~\cite{LamportBGP} и описывала ситуацию, в которой группа генералов византийской армии вместе со своими войсками расположена вокруг вражеского лагеря. Они могут общаться только через посланников и должны договориться об общем плане действий: отступать или нападать. Однако один или несколько из них могут быть предателями, которые попытаются запутать остальных, посылая при этом ложные данные. Проблема заключается в том, чтобы найти алгоритм, который обеспечит достижение согласия среди преданных генералов. 

Формальное описание данной задачи заключается в том, что надо найти метод достижения консенсуса в распределенной системе, при условии, что некоторые ее элементы могут работать неверно. В качестве входных данных алгоритму подаются исправные и неисправные узлы системы. Выходными данными является единое решение, принятое для всех узлов в зависимости от функций, выполняемых самой системой.

\FloatBarrier
\imgw{1\textwidth}{01_A0}{Формализация задачи византийских генералов в нотации IDEF0}
\FloatBarrier

\section{Область применения}

Впервый задача византийских генералов была сформулирована в 1982 году в статье Л. Лэмпорт, Р. Шостак и М. Пиз~\cite{LamportBGP}. В данной работе поднималась проблема взаимодействия элементов распределенной системы между собой, которая иллюстрировалась на примере генералов, которым нужно договориться между собой о плане наступления.

На основе поставленной задачи в исследовании Барбары Лисков и Мигель Кастро~\cite{Castro2002} был предложен алгоритм практической византийской отказоустойчивость, актуальность которого заключалась в необходимости высоконадежных систем, которые обеспечивают корректное обслуживание без сбоев. 

Следующим важным событием являлась публикация Сатоши Накамото в 2008 году идеи алгоритма достижения консенсуса в децентрализованных системах не требующих аутентификации ее узлов~\cite{Bitcoin}. Данная реализация алгоритма доказательста работы, положила начало развития блокчейн-технологии.

В современном мире задача византийских генералов является достаточной актуальрой и рассматривается при реализации технологии распределённых реестров~\cite{reestr}, одной из которых является блоекчейн, при созданиии распределенных операционных систем~\cite{lec} или интернет протоколов~\cite{icc}.


%При создании распределенных систем важным является организации взаимодействия ее одновременно работающих элементов между собой и принятия ими единого решения, то есть достижения консенсуса, даже в ситуации, когда некоторые узлы системы начинают работать неверно, злонамеренно или в случае поломки~\cite{distrDataProc}.


%Одним из примеров систем, в которых необходимо достижения византийской отказоустойчивости является:
%блокчейн, который основывается на распределенном хранении данных, и так как в нем нет основного узла отвечающего за все операции, данной системе необходимо использование протоколов достижения консенсуса~\cite{blockchain}.


%Алгоритмы достижения консенсуса в распределенных системах, основанные на решении задачи византийских генералов рассматриваются в нескольких областях:
%\begin{itemize}
%	\item в блокчейн-технологии~\cite{blockchain};
%	\item в распределенных операционных системах, устойчивых к отказам~\cite{lec};
%\end{itemize}
%
%Блокчейн-технология представляет собой выстроенную по определённым
%правилам непрерывную последовательную цепочку блоков, содержащих информацию
%
%возможность свободного подключения нового узла к сети без получения каких-либо разрешений.
%
%В блокчейн сетях нет центрального органа, который обеспечивал бы одинаковую работу на удалённых узлах. Таким образом, имеется потребность в протоколах обеспечения консенсуса между распределёнными узлами


%\clearpage
%\section*{Вывод из аналитической части}

\clearpage
