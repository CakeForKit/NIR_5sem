\chapter{Методы решения задачи византийских генералов}

%В данном разделе представлены математические основы и схема алгоритма обратной трассировки лучей, функциональная модель и структура программного обеспечения. 
%Так же будут описаны используемые структуры данных и структура реализуемого программного обеспечения.

Существует большое количество алгоритмов достижения консенсуса в распределенных системах, обладающих византийской отказоустойчивостью. В данной научной работе рассматриваются основные из них, которые легли в основу создания всех остальных.


\section{Практическая византийская отказоустойчивость}

Метод практической византийской отказоустойчивости (Practical Byzantine Fault Tolerance, PBFT), предложенный в 2002 году Барбарой Лисков и Мигель Кастро~\cite{Castro2002}, является алгоритмом основанном на передаче подписанных сообщений между репликами. 

Подпись сообщений производится с использованием методов симметричной криптографии, для аутентификации самих узлов в системе.


При условии наличия в системе, состоящей из $3k + 1$ узлов, не более $k$ неисправных, алгоритм гарантирует ее верную работу.~\cite{Tanenbaum}. 


Существует не малое количество модификаций данного алгоритма: метод sdBFT (stake distributed ВЕТ)~\cite{algsBch}, который позволяет увеличить количество узлов, участвующих в достижении консенсуса без потери скорости поступления транзакций; или семейство протоколов Internet Computer Consensus (ICC)~\cite{icc}, основным достоинство которого является скорость работы алгоритма равная скорости фактической сетевой задержки, а не с некоторой верхней границей сетевой задержки, в случае когда лидер честен.


\section{Доказательство выполнения работы}

Метод доказательства выполнения работы (Proof of Work, PoW), разработанный Сатоши Накамото~\cite{Bitcoin},заключается в требовании от узла вычисления значения, <<nonce>>, которое при хэшировании начинается с определенного количества нулевых битов, то есть вычислить значение хеша заголовка блока, который при этом может регулярно меняться. 

После получения искомого значения остальным элементам сети надо взаимно подтвердить корректность полученного хеша. Таким образом транзакции, хранящиеся в новом блоке, проверяются в случае мошенничества. 

При последовательном решении поставленной задачи честными узлами образуется длинная цепочка связных блоков. И для изменения ее элементов необходимо будет повторить проверку его работоспособности и всех блоков после него, а затем догнать и превзойти работу честных узлов.

Алгоритм PoW обеспечивает безопасность системы, только когда мощность одного вычислительного узла не превышает половины мощностей всей сети~\cite{blockchain}.

Из-за необходимости произведения огромного количества математических операций в данном методе, а вследствие и большого потребления электроэнергии, был придуман более эффективный и экономичный алгоритм Proof of Stake (PoS), в котором для подключения к сети блока необходимо доказать владение определенным количеством валюты.
%Одним из решений может является
%
%Решение большинства будет представлено самой длинной цепочкой, в которую вложено наибольшее количество усилий для проверки работоспособности. Если большая часть ресурсов процессора контролируется честными узлами, честная цепочка будет расти быстрее всех и опередит любые конкурирующие цепочки. А чтобы изменить предыдущий блок, злоумышленнику пришлось бы повторить проверку работоспособности этого блока и всех блоков после него, а затем догнать и превзойти работу честных узлов.
%
%
%Чтобы компенсировать увеличение скорости работы оборудования и изменение интереса к запущенным узлам с течением времени, сложность проверки работоспособности определяется скользящим средним, рассчитанным на среднее количество блоков в час. Если они генерируются слишком быстро, сложность возрастает.


\section{Сравнение методов решения задачи византийских генералов}

В качестве сравнения рассмотренный методов решения задачи византийских генералов были выбраны критерии:
\begin{itemize}
	\item анонимность узлов, так как в некоторых распределенных системах, например, в открытых блокчейн сетях, направленных на децентрализацию, требуется конфиденциальность элементов;
	\item уровень устойчивости алгоритма, который определяется отношением неисправных элементов системы($k$) к исправным($n$);
%	\item доля злонамеренных элементов системы, которым может противостоять алгоритм, так как от этого зависит степень защищенности, которую предоставляет алгоритм;
\end{itemize}

\begin{longtable}{|
		>{\centering\arraybackslash}m{.5\textwidth - 2\tabcolsep}|
		>{\centering\arraybackslash}m{.25\textwidth - 2\tabcolsep}|
		>{\centering\arraybackslash}m{.25\textwidth - 2\tabcolsep}|
	}
	\caption{Сравнение методов решения задачи византийских генералов}\label{tbl:resTest} \\\hline
	& PBFT & PoW  \\\hline
	\endfirsthead
	\caption*{Продолжение таблицы~\ref{tbl:log} } \\\hline
	& PBFT & PoW  \\\hline \\\hline                    
	\endhead
	\endfoot
	Анонимность узлов & - & + \\\hline
	Уровень устойчивости алгоритма & $3k + 1 \leq n$ & $2k \leq n$ \\\hline
	%	Functions & 220 & 463 & 47.5 \% \\\hline
\end{longtable}


При сравнении рассматриваемых методов, можно сделать вывод, что алгоритм PoW, основанный на доказательстве, хоть и сохраняет конфиденциальность своих узлов, но обладает меньшей отказоустойчивостью, чем алгоритм PBFT, основанный на подписях и опросе узлов.

\clearpage
